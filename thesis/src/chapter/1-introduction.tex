\chapter{Einleitung}\label{ch:introduction}
\section{Motivation}
\begin{itemize}
    \item Spielprojekt mit schlechtem Speichersystem 
    \item Unity hat kein richtiges Speichersystem angeboten, musste alles selber machen
\end{itemize}

\section{Forschungsfrage}
Die Hauptsächliche Forschungsfrage dieser Arbeit ist, wie besonders effizient ein Speicher- und Ladesystem aufgestellt werden kann. Also wie möglichst schnell ein Spielstand beim Spielstart geladen und anschließend während der Spielphase noch benötigte Daten des gespeicherten Spielstandes geladen und neue Ereignisse gespeichert werden können. Dabei sollte beachtet werden, dass Videospiele verschiedene Arten von Daten und Spielwelten haben. Durch die Forschung zu dieser Thematik sollten also auch die Stärken und Schwächen von verschiedenen Strategien ausgearbeitet werden, um ihren Einsatzbereich besser zu verstehen. Außerdem sollten Strategien auf ihrer Anpassungsfähigkeit getestet werden. In manchen Videospielen können sich die Spieldaten stark verändern und das Speicher- und Ladesystem sollte in jedem Szenario problemlos laufen und die Erfahrung des Spieles nicht durch Verlangsamung beeinträchtigen. Die Strategien, die einfach umzusetzen sind und trotzdem effizient laufen, sollten auch betrachtet werden. Das Entwickeln eines Spieles kostet viel Zeit, ein Speicher- und Ladesystem sollte also nicht enorm die Arbeitszeit erhöhen, damit Entwickler sich auf auf andere Spieleinhalte fokussieren können. Ein gutes Speicher- und Ladesystem gehört nämlich\todo{synonym} nicht zu den Hauptfeatures\todo{synonym} eines Spieles, sondern wird nicht auffallen, da es hauptsächlich im Hintergrund läuft. Jedoch benötigen trotzdem fast alle Videospiele ein Speichersystem.  

Die nächste Frage ist, ob moderne Game Engines Funktionen anbieten, die das Erstellen eines Speicher- und Ladesystems erleichtern oder sogar die meiste Arbeit dafür abnehmen. Verwenden die Game Engines dabei irgendwelche Techniken, zum Speichern und Laden des Spielstandes? Vielleicht gibt es auch in der Spieleindustrie bereits Standards zum Angehen dieser Thematik. Wie gehen moderne Spiele das Speichern und Laden des Spielstandes an?

% \begin{itemize}
%     \item Wie kann besonders effizient ein Spielstand geladen und während der Spielzeit Daten des alten Spielstandes geladen und die neuen gespeichert werden?
%     \item Gibt es Strategien, die bei manchen Arten von Spielen vorteilhafter sind?
%     \item Gibt es Strategien, die besonders anpassungsfähig sind und in vielen verschiedenen Situationen effizient arbeiten?
%     \item Ist es möglich eine Strategie zu finden, die möglichst effizient läuft und einfach umzusetzen ist, damit Entwickler sich auf Inhalte des Spieles fokussieren können?
%     \item Verwenden Game Engines Techniken oder bieten sie Funktionen zum Speichern und Laden der Spieldaten an?
%     \item Hat die Spieleindustrie einen Standard beim Speichern und Laden des Spielstandes?
%     \item Wie gehen populäre Spiele das Speichern und Laden des Spielstandes an?
% \end{itemize}



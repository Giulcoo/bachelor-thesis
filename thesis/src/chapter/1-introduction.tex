\chapter{Einleitung}\label{ch:introduction}
Videospiele werden in der ganzen Welt immer populärer. Von Jahr zu Jahr gibt es immer mehr Menschen, die aktiv Videospiele spielen.\cite{explodingtopicsManyGamers} Nicht nur die Anzahl der Spieler steigt, auch die Branche wird immer größer. In den letzten Jahren waren die Umsätze der Videospielbranche in hunderten von Milliarden US-Doller Größe und steigen auch konstant weiter.\cite{statistaUmsatzVideogames} Trotz der Popularität und des Wachstums der Videospielbranche sind im Allgemeinen nicht viele Standards, wie zum Beispiel bei der Webentwicklung, entstanden. Viele Teams, die Spiele entwickeln, arbeiten sehr unterschiedlich von anderen beim Erstellen ihrer Spiele. Dies liegt teilweise auch da dran, dass wenige Videospiele für die Öffentlichkeit dokumentiert wurden und das aufgebaute Wissen und Techniken nicht weitergegeben werden. Dieses Problem scheint auch bei der Entwicklung eines Speicher- und Ladesystems der Fall zu sein, obwohl jedes Spiel diese Systeme benötigt. Die Daten in Videospielen werden auch immer größer, weshalb die Systeme zum Sichern des Spielzustandes immer effizienter werden müssen, damit sie zuverlässig und schnell arbeiten können.

\section{Motivation}
Speicher- und Ladesysteme werden in fast allen Videospielen benötigt. Es gibt in den meisten Spielen irgendwelche Arten von Daten eines Spielstandes, die gesichert werden müssen. Ein gutes Speichersystem hat auch die Eigenschaft, dass es nicht auffällt und stets dafür sorgt, dass kein Spielstand verloren geht. Es sollte auch nicht das Spiel verlangsamen, weshalb es so effizient wie möglich laufen sollte, da die Anzahl der zu speichernden Spieldaten sehr groß werden kann. Bei Game Engines wie Unity wird auch kein fertiges Speicher- und Ladesystem bereitgestellt, obwohl Unity sehr viele Funktionen, die in vielen Spielen gebraucht werden, zur Verfügung stellt. Es scheint also eine komplexe Thematik zu sein, bei der vielleicht noch keine richtigen Standards existieren. 

% \begin{itemize}
%     \item Speicher- und Ladesysteme in fast jedem Spiel benötigt
%     \item Gute Speicher- und Ladesysteme fallen nicht auf -> Laufen im Hintergrund und verlangsamen nicht den Rest des Spieles
%     \item Game Engines wie Unity stellen kein Speicher- und Ladesystem bereit, obwohl das fast jedes Spiel braucht und viele andere Features, die jedes Spiel braucht, vorhanden sind
%     \item Scheint ein komplexes Thema zu sein, welches noch keine richtige Standards hat 
% \end{itemize}

\section{Forschungsfrage}
Die hauptsächliche Forschungsfrage dieser Arbeit ist, wie besonders effizient ein Speicher- und Ladesystem aufgestellt werden kann. Also, wie möglichst schnell ein Spielstand beim Spielstart geladen und anschließend während der Spielphase noch benötigte Daten des gespeicherten Spielstandes geladen und neue Ereignisse gespeichert werden können. Dabei sollte beachtet werden, dass Videospiele verschiedene Arten von Daten und Spielwelten haben. Durch die Forschung zu dieser Thematik sollten also auch die Stärken und Schwächen von verschiedenen Strategien ausgearbeitet werden, um ihren Einsatzbereich besser zu verstehen. Außerdem sollten Strategien auf ihre Anpassungsfähigkeit getestet werden. In manchen Videospielen können sich die Spieldaten stark verändern und das Speicher- und Ladesystem sollte in jedem Szenario problemlos laufen und die Erfahrung des Spieles nicht durch Verlangsamung beeinträchtigen. Die Strategien, die einfach umzusetzen sind und trotzdem effizient laufen, sollten auch betrachtet werden. Das Entwickeln eines Spieles kostet viel Zeit, ein Speicher- und Ladesystem sollte also nicht enorm die Arbeitszeit erhöhen, damit Entwickler sich auf andere Spieleinhalte fokussieren können. Ein gutes Speicher- und Ladesystem gehört nicht zu den Hauptmerkmalen eines Spieles, welches es von anderen unterscheidet. Das System wird auch nicht auffallen, da es hauptsächlich im Hintergrund läuft. Jedoch benötigen trotzdem fast alle Videospiele ein Speichersystem.  

Die nächste Frage ist, ob moderne Game Engines Funktionen anbieten, die das Erstellen eines Speicher- und Ladesystems erleichtern oder sogar die meiste Arbeit dafür abnehmen. Verwenden die Game Engines dabei irgendwelche Techniken zum Speichern und Laden des Spielstandes? Vielleicht gibt es auch in der Spieleindustrie bereits Standards zum Angehen dieser Thematik. Wie gehen moderne Spiele das Speichern und Laden des Spielstandes an?

% \begin{itemize}
%     \item Wie kann besonders effizient ein Spielstand geladen und während der Spielzeit Daten des alten Spielstandes geladen und die neuen gespeichert werden?
%     \item Gibt es Strategien, die bei manchen Arten von Spielen vorteilhafter sind?
%     \item Gibt es Strategien, die besonders anpassungsfähig sind und in vielen verschiedenen Situationen effizient arbeiten?
%     \item Ist es möglich eine Strategie zu finden, die möglichst effizient läuft und einfach umzusetzen ist, damit Entwickler sich auf Inhalte des Spieles fokussieren können?
%     \item Verwenden Game Engines Techniken oder bieten sie Funktionen zum Speichern und Laden der Spieldaten an?
%     \item Hat die Spieleindustrie einen Standard beim Speichern und Laden des Spielstandes?
%     \item Wie gehen populäre Spiele das Speichern und Laden des Spielstandes an?
% \end{itemize}



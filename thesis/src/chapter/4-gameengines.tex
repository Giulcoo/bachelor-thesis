\chapter{Game Engines}\label{ch:gameengines}
In diesem Kapitel soll die Theorie und Algorithmen aus den vorherigen Kapiteln 
getestet werden. Manche Game Engines bieten auch eigene Speichersysteme oder
Tools dafür, diese können hier auch getestet werden.

Gleiche Szene in verschiedenen Game Engines testen. 
Verschiedene Speichersysteme ausprobieren und wenn vorhanden über die von den Game Engines erzählen und ausprobieren.

Zu testende Game Engines: Unity, Unreal Engine, Godot

Szene zum testen soll sehr viele Daten zu speichern haben (Große Map und viele unterschiedliche Informationen)

\section{Unity}
\url{https://blog.unity.com/games/persistent-data-how-to-save-your-game-states-and-settings}


\subsection{PlayerPrefs}
\begin{itemize}
    \item Nur zum Speichern von string, float, integer
\end{itemize}

\url{https://docs.unity3d.com/ScriptReference/PlayerPrefs.html}\\
\url{https://blog.logrocket.com/why-should-shouldnt-save-data-unitys-playerprefs/}

\subsection{JSON}
\begin{itemize}
    \item JsonUtility (Von Unity. Hat Einschränkungen, ist dafür aber schnell. Anwendungsbereich ausarbeiten)
    \item Newtonsoft (Unity Package)
    \item LitJSON (.Net Library)
\end{itemize}

\url{https://docs.unity3d.com/2020.1/Documentation/Manual/JSONSerialization.html}\\
\url{https://www.newtonsoft.com/json}\\
\url{https://www.newtonsoft.com/json/help/html/Performance.htm}
\url{https://litjson.net/}

\subsection{BinaryFormatter}
Soll nicht so gut sein. Kurz erwähnen was BinaryFormatter macht und warum es nicht benutzt
werden soll.

\url{https://learn.microsoft.com/en-us/dotnet/api/system.runtime.serialization.formatters.binary.binaryformatter?view=net-7.0}\\
\url{https://learn.microsoft.com/en-us/dotnet/standard/serialization/binaryformatter-security-guide}\\
\url{https://www.mongodb.com/developer/products/realm/saving-data-in-unity3d-using-binary-reader-writer/}

\subsection{StreamWriter}
\url{https://learn.microsoft.com/en-us/dotnet/api/system.io.streamwriter?view=net-7.0}

\section{Unreal Engine}
\subsection{SaveGame Objekt}
https://docs.unrealengine.com/4.26/en-US/InteractiveExperiences/SaveGame/

\subsection{JSON}
\url{https://docs.unrealengine.com/4.27/en-US/API/Runtime/Json/Serialization/FJsonSerializer/}\\
\url{https://docs.unrealengine.com/4.27/en-US/API/Runtime/Json/Dom/FJsonObject/}

\section{Godot}

\subsection{JSON}
Macht viel Sinn, weil GodotScript Dictionaries hat, die Sachen wie im JSON Stil speichern 
können.

\url{https://docs.godotengine.org/en/stable/tutorials/io/saving_games.html}\\
\url{https://docs.godotengine.org/en/stable/tutorials/io/saving_games.html#json-vs-binary-serialization}
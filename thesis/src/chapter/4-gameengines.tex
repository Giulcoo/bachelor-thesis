\chapter{Game Engines}\label{ch:gameengines}
In diesem Kapitel soll die Theorie und Algorithmen aus den vorherigen Kapiteln 
getestet werden. Manche Game Engines bieten auch eigene Speichersysteme oder
Tools dafür, diese können hier auch getestet werden.

Gleiche Szene in verschiedenen Game Engines testen. 
Verschiedene Speichersysteme ausprobieren und wenn vorhanden über die von den Game Engines erzählen und ausprobieren.

Zu testende Game Engines: Unity, Unreal Engine, Godot

Szene zum testen soll sehr viele Daten zu speichern haben (Große Map und viele unterschiedliche Informationen)
%--------------------------------------------------------------------------


%--------------------------------------------------------------------------
\section{Unity}
Unity Engine grob vorstellen und erklären
\url{https://blog.unity.com/games/persistent-data-how-to-save-your-game-states-and-settings}


\subsection{PlayerPrefs}
\begin{itemize}
    \item Nur zum Speichern von string, float, integer
\end{itemize}

\url{https://docs.unity3d.com/ScriptReference/PlayerPrefs.html}\\
\url{https://blog.logrocket.com/why-should-shouldnt-save-data-unitys-playerprefs/}

\subsection{JSON}
\begin{itemize}
    \item JsonUtility (Von Unity. Hat Einschränkungen, ist dafür aber schnell. Anwendungsbereich ausarbeiten)
    \item Newtonsoft (Unity Package)
    \item LitJSON (.Net Library)
\end{itemize}

\url{https://docs.unity3d.com/2020.1/Documentation/Manual/JSONSerialization.html}\\
\url{https://www.newtonsoft.com/json}\\
\url{https://www.newtonsoft.com/json/help/html/Performance.htm}
\url{https://litjson.net/}

\subsection{BinaryFormatter}
Soll nicht so gut sein. Kurz erwähnen was BinaryFormatter macht und warum es nicht benutzt
werden soll.

\url{https://learn.microsoft.com/en-us/dotnet/api/system.runtime.serialization.formatters.binary.binaryformatter?view=net-7.0}\\
\url{https://learn.microsoft.com/en-us/dotnet/standard/serialization/binaryformatter-security-guide}\\
\url{https://www.mongodb.com/developer/products/realm/saving-data-in-unity3d-using-binary-reader-writer/}

\subsection{StreamWriter}
\url{https://learn.microsoft.com/en-us/dotnet/api/system.io.streamwriter?view=net-7.0}

\subsection{Easy Save}
\url{https://docs.moodkie.com/product/easy-save-3/}\\
\url{https://assetstore.unity.com/packages/tools/utilities/easy-save-the-complete-save-data-serializer-system-768}
%--------------------------------------------------------------------------


%--------------------------------------------------------------------------
\section{Unreal Engine}
Unreal Engine grob vorstellen und erklären

\subsection{SaveGame Objekt}
Erklären was das ist, was benutzt wird und ein kleines Beispiel zeigen.

\url{https://docs.unrealengine.com/4.26/en-US/InteractiveExperiences/SaveGame/}\\
\url{https://www.tomlooman.com/unreal-engine-cpp-save-system/}

\subsection{JSON}
\url{https://docs.unrealengine.com/4.27/en-US/API/Runtime/Json/Serialization/FJsonSerializer/}\\
\url{https://docs.unrealengine.com/4.27/en-US/API/Runtime/Json/Dom/FJsonObject/}

\subsection{Save Extension}
\url{https://www.unrealengine.com/marketplace/en-US/product/save-extension}\\
\url{https://piperift.com/SaveExtension/#/}
%--------------------------------------------------------------------------


%--------------------------------------------------------------------------
\section{Godot}
Godot Engine grob vorstellen und erklären

\subsection{JSON}
Macht viel Sinn, weil GodotScript Dictionaries hat, die Sachen wie im JSON Stil speichern 
können. JSON wird dann zu einem Dictionary umgeformt und ein Dictionary in JSON

\url{https://docs.godotengine.org/en/stable/tutorials/io/saving_games.html}\\
\url{https://docs.godotengine.org/en/stable/tutorials/io/saving_games.html#json-vs-binary-serialization}\\
\url{https://docs.godotengine.org/en/3.0/tutorials/io/saving_games.html}

\subsection{Universal Save/Load System}
\url{https://godotengine.org/asset-library/asset/1655}\\
\url{https://stupidratstudio.github.io/thoth/}
\chapter{Online}\label{ch:online}
Welche Möglichkeiten gibt es Spielerdaten online abzuspeichern (zum Beispiel um den gleichen Spielstand über mehrere Geräte zu erreichen oder um ein Cloud Save anzubieten)

\section{Databases}
Welche Datenbanken kann man einrichten, um selber ein Cloud Save zu erstellen. Und Frage SQL vs NoSQL für Videospiel-Daten herausfinden (Vielleicht eigenes Kapitel: Speicherart document database vs relational database). Vielleicht ist es am Ende nur die Frage, SQL Performance vs NoSQL optimalere Art Daten dazustellen (für Video Spielen).

\url{https://redis.com/solutions/industries/gaming/}\\
\url{https://gamedev.stackexchange.com/questions/5316/nosql-is-it-a-valid-option-for-web-based-game}

\subsection{NoSQL}
\subsubsection{MongoDB}

\url{https://kushankpatel7.medium.com/a-database-for-video-games-794b243f0758}\\
\url{https://www.linkedin.com/pulse/mongodb-gaming-udit-sharma}

\subsection{SQL}
\subsubsection{PostgreSQL}
\begin{itemize}
    \item Open source object-relational database system
    \item Verschiedene Datentypen
    \begin{itemize}
        \item Primitive
        \item Structured (Arrays, UUID, ...)
        \item Document (JSON, XML, ...)
        \item Geometrisch (Punkt, Linie, Polygon, ...)
    \end{itemize}
    \item Performant (Beispiele zeigen)
    \item Disaster Recovery
    \item Sicherheit (Mehr Infos darüber wäre nice)
\end{itemize}

\url{https://www.postgresql.org/about/}\\
\url{https://www.amazingcto.com/postgres-for-everything/}\\
\url{https://kinsta.com/de/wissensdatenbank/was-ist-postgresql/}\\
\url{https://www.ibm.com/de-de/topics/postgresql}

\section{Cloud Services}
Welche bereits fertige Cloud Services gibt es

\subsection{Unity Cloud Save} 
Vor- und Nachteile oder Einschränkungen\\

\begin{itemize}
    \item Cloud save data of game made in unity
    \item Cross-device Funktionalität
    \item Sicherer Speichersystem
    \item Max 255 Charakter pro Slotname
    \item Max 2000 Data Slots pro Spieler
    \item Max 5 MB Speicher pro Spieler über alle seine Slots
    \item JSON Daten
\end{itemize}

\url{https://unity.com/products/cloud-save}

\subsection{Azure}
Azure bietet Cosmos DB an \\
\begin{itemize}
    \item Schnell und skalierbar
    \item Autoscale, damit sich die DB an unvorhersehbaren traffic anpasst
    \item Unterstützt NoSQL, MongoDB, PostgreSQL, Apache Cassandra
\end{itemize}
\url{https://azure.microsoft.com/de-de/products/cosmos-db}

\subsection{AWS}
Skalierbare Datenbanken, einfach Einzurichten und mit hoher Leistung
\begin{itemize}
    \item Amazon Aurora für MySQL und PostgreSQL \url{https://aws.amazon.com/de/rds/aurora/?did=ap_card&trk=ap_card}
    \item Amazon DynamoDB für schnelle NoSQL-Schlüssel-Werte-Datenbank mit Leistung im einstelligen Millisekundenbereich \url{https://aws.amazon.com/de/dynamodb/?did=ap_card&trk=ap_card}
    \item Amazon DocumentDB für JSON \url{https://aws.amazon.com/de/documentdb/?did=ap_card&trk=ap_card}
\end{itemize}

\url{https://aws.amazon.com/de/gametech/databases/}

\subsection{Firebase}

\begin{itemize}
    \item NoSQL Datenbank von Google
    \item Cross-device
    \item Nur für iOS, tvOS oder Android
    \item Wenn offline benutzt werden Veränderungen zwischengespeichert
\end{itemize}

\url{https://firebase.google.com/products/realtime-database}

\subsection{Atlas}

\begin{itemize}
    \item JSON Datenbank
    \item Cross-device
    \item Sicherheit
\end{itemize}

\url{https://www.mongodb.com/atlas}


\subsection{Cloud Spanner}

\url{https://cloud.google.com/spanner?hl=de}\\
\url{https://cloud.google.com/spanner/docs/best-practices-gaming-database?hl=de}
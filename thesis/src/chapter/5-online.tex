\chapter{Online}\label{ch:online}
Welche Möglichkeiten gibt es Spielerdaten online abzuspeichern (zum Beispiel um den gleichen Spielstand über mehrere Geräte zu erreichen oder um ein Cloud Save anzubieten)

\section{Databases}
Welche Datenbanken kann man einrichten, um selber ein Cloud Save zu erstellen. Und Frage SQL vs NoSQL für Videospiel-Daten herausfinden (Vielleicht eigenes Kapitel: Speicherart document database vs relational database). Vorteil von NoSQL ist, dass Objekte direkt zu JSON Dokumenten umgewandelt werden können

\subsection{NoSQL}
\subsubsection{MongoDB}

\subsection{SQL}
\subsubsection{PostgreSQL}

\section{Cloud Services}
Welche bereits fertige Cloud Services gibt es

\subsection{Unity Cloud Save} 
Vor- und Nachteile oder Einschränkungen\\

\begin{itemize}
    \item Cloud save data of game made in unity
    \item Cross-device Funktionalität
    \item Sicherer Speichersystem
    \item Max 255 Charakter pro Slotname
    \item Max 2000 Data Slots pro Spieler
    \item Max 5 MB Speicher pro Spieler über alle seine Slots
    \item JSON Daten
\end{itemize}

\url{https://unity.com/products/cloud-save}

\subsection{Azure}
Azure bietet Cosmos DB an \\
\begin{itemize}
    \item Schnell und skalierbar
    \item Autoscale, damit sich die DB an unvorhersehbaren traffic anpasst
    \item Unterstützt NoSQL, MongoDB, PostgreSQL, Apache Cassandra
\end{itemize}
\url{https://azure.microsoft.com/de-de/products/cosmos-db}

\subsection{AWS}
Skalierbare Datenbanken, einfach Einzurichten und mit hoher Leistung
\begin{itemize}
    \item Amazon Aurora für MySQL und PostgreSQL \url{https://aws.amazon.com/de/rds/aurora/?did=ap_card&trk=ap_card}
    \item Amazon DynamoDB für schnelle NoSQL-Schlüssel-Werte-Datenbank mit Leistung im einstelligen Millisekundenbereich \url{https://aws.amazon.com/de/dynamodb/?did=ap_card&trk=ap_card}
    \item Amazon DocumentDB für JSON \url{https://aws.amazon.com/de/documentdb/?did=ap_card&trk=ap_card}
\end{itemize}

\url{https://aws.amazon.com/de/gametech/databases/}

\subsection{Firebase}

\begin{itemize}
    \item NoSQL Datenbank von Google
    \item Cross-device
    \item Nur für iOS, tvOS oder Android
    \item Wenn offline benutzt werden Veränderungen zwischengespeichert
    \item 
\end{itemize}

\url{https://firebase.google.com/products/realtime-database}

\subsection{Atlas}

\begin{itemize}
    \item JSON Datenbank
    \item Cross-device
    \item Sicherheit
\end{itemize}

\url{https://www.mongodb.com/atlas}
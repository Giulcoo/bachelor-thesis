\chapter{Algorithmen}\label{ch:algorithmen}
Rechtschreib Extension für VS Code!!!!!!!!!!!

Verschiedene Theorien von Speicher und Ladesysteme ausarbeiten und Zeiten 
messen (Mit Java und JSON erstmal). Verschiedene Spielarten und Speichersysteme
für diese ausarbeiten (Chunk Systeme machen vielleicht nicht überall Sinn, aber 
allgemein Daten aufteilen müsste gut sein)

\section{Speichersysteme}
Erstmal Theorie von Speichersystemen und Bezug auf Arten von Videospiel-Daten.

Theorie durch Daten/einfaches Spielkonzept mit Java und JSON testen:\\
-Spieler (HP, LVL, Ausrüstung, Position, Rotation, ...)\\
-Gegner (HP, LVL, Ausrüstung, Position, Rotation, ...)\\
-Ausrüstung (Beschreibung/ID, Verteidigung, Angriff)\\
-Hindernisse (Beschreibung/ID, Position, Rotation)\\
-Items (Beschreibung/ID, Position, Rotation)\\
-> Klassendiagramm zeigen, um Daten des Spieles zu visualisieren

\subsection{Varianten von Daten in Videospielen}
Statische Daten (Maps, Texturen, etc.) und sich ändernde Daten (Spielerdaten und 
Daten zu dem Spielstand). Fokus dieser Arbeit sind nicht die statischen Daten!

\subsection{Aufteilung der Daten}
\subsection{Kodierung der Daten}
Eigentlich auch eine Art der Komprimierung der Daten, wenn richtig gemacht. 

\subsection{Komprimierung der Daten}
Zippen der Daten

\subsection{Speicherprozess}
Effizientes schreiben in (JSON) Datenbanken
\subsection{Laufzeiten der Speicherstrategien}

\section{Ladesysteme}
Erstmal Theorie vom Laden der Spieldaten erarbeiten. Danach mit Spielkonzept
von 2.1 testen und Laufzeiten messen.

\subsection{Lazy Loading}
Weiß nicht, ob das der richtige Begriff dafür ist, aber hier soll über das prozedulare 
Laden der Daten geschrieben werden (nicht alles auf einmal laden, sondern erstmal nur das
nötigste und dann Stück für Stück den Rest laden). 

\subsection{Laufzeiten der Ladestrategien}
Laufzeiten vom Laden messen. Vielleicht läuft das Laden mit manchen Speicherstrategien 
besser als mit anderen. Außerdem ist die Performance nicht nur an Zeit zu messen, sondern
an Ressourcen, die vom Computer benötigt werden. Vielleicht gibt es Lade Strategien, die
leichtgewichtet laufen können.
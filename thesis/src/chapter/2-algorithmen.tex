\chapter{Algorithmen}\label{ch:algorithmen}
Forschungsfrage überlegen (Warum nicht DB zum Speichern von Spielerdaten)

Verschiedene Theorien von Speicher und Ladesysteme ausarbeiten und Zeiten 
messen (Mit Java und JSON erst mal). Verschiedene Spielarten und Speichersysteme
für diese ausarbeiten (Chunk Systeme machen vielleicht nicht überall Sinn, aber 
allgemein Daten aufteilen müsste gut sein)
%--------------------------------------------------------------------------


%--------------------------------------------------------------------------
\section{Speichersysteme}
Erstmal Theorie von Speichersystemen und Bezug auf Arten von Videospiel-Daten.

Theorie durch Daten/einfaches Spielkonzept mit Java und JSON testen:\\
-Spieler (HP, LVL, Ausrüstung, Position, Rotation, ...)\\
-Gegner (HP, LVL, Ausrüstung, Position, Rotation, ...)\\
-Ausrüstung (Beschreibung/ID, Verteidigung, Angriff)\\
-Hindernisse (Beschreibung/ID, Position, Rotation)\\
-Items (Beschreibung/ID, Position, Rotation)\\
-> Klassendiagramm zeigen, um Daten des Spieles zu visualisieren

Wichtiger unterschied zwischen ersten Speichern und das Speichern Danach
(beim ersten Mal muss die Grundstruktur aufgebaut werden und die random
erstellte Map gespeichert werden)
%--------------------------------------------------------------------------


%--------------------------------------------------------------------------
\subsection{Daten in Videospielen}
Statische Daten (Maps, Texturen, etc.) und sich ändernde Daten (Spielerdaten und 
Daten zu dem Spielstand). Fokus dieser Arbeit sind nicht die statischen Daten!

Typen von Daten:\\
\begin{itemize}
    \item Statische Daten
    \begin{itemize}
        \item Grafikdaten
        \item Audiotechnische Daten
        \item Level- oder Kartendaten
    \end{itemize}
    \item Dynamische Daten
    \begin{itemize}
        \item Spielstand
        \item Benutzerdaten
    \end{itemize}
\end{itemize}
%--------------------------------------------------------------------------


%--------------------------------------------------------------------------
\subsection{Spielphasen}
Verschiedene Phasen ansprechen, die interessant für's Speichern und Laden sind.\\
1. Neues Spiel laden oder altes Spiel laden\\
2. Events in Spiel (Spieler bewegt sich, Item spawned, ...) -> Events speichern\\
3. Chunk Laden 
%--------------------------------------------------------------------------


%--------------------------------------------------------------------------
\subsection{Delta basierte Speicherung}
Speichern nur von den veränderten Daten durch tracken der Änderungen
\begin{itemize}
    \item Neue Elementen
    \item Elemente verändern sich
    \item Elemente werden gelöscht
\end{itemize}
Gespeicherte Daten müssen diese drei Veränderungen berücksichtigen
%--------------------------------------------------------------------------


%--------------------------------------------------------------------------
\subsection{Aufteilung der Daten}
Chunking\\
\begin{itemize}
    \item Aufteilung der Mengen an Daten unter den Chunks
    \item Datengröße kann sich stark verändern, Chunk System muss sich anpassen
    \item Wenn Daten zu viele werden mehr Chunks
    \item Wenn Daten zu wenig werden weniger Chunks
    \item Chunk System nach Area, damit Laden effizienter läuft 
    (Nur Chunks mit seinen Elementen laden, wenn dieser in Nähe ist)
\end{itemize}

Sharding (Mehr was für Kapitel Datenbanken)
%--------------------------------------------------------------------------


%--------------------------------------------------------------------------
\subsection{Komprimierung der Daten}
Zippen der Daten\\
\begin{itemize}
    \item Viel zippen vs. kleine Dateien (Kleine Chunks)
    \item Bringt lokales Zippen überhaupt was oder nur wenn man mit Server arbeitet?
\end{itemize}

Kürzere property Namen\\
Null Werte ausschließen\\
Sowas wie Vectoren (x,y,z,...) als ein String speichern?
%--------------------------------------------------------------------------


%--------------------------------------------------------------------------
\subsection{Kodierung der Daten}
Eigentlich auch eine Art der Komprimierung der Daten, wenn richtig gemacht.\\
JSONH zum Beispiel (\url{https://stackoverflow.com/questions/11160941/is-it-worth-the-effort-to-try-to-reduce-json-size})

\subsection{Serialisieren}
Effizientes schreiben in (JSON) Datenbanken.\\\\
Arten um mit JSON zu arbeiten:
\begin{itemize}
    \item Streaming API
    \item Tree Model
    \item Data Binding
\end{itemize}
(Siehe \url{https://www.tutorialspoint.com/jackson/jackson_overview.htm})
%--------------------------------------------------------------------------


%--------------------------------------------------------------------------
\subsection{Effizienz der Speicherstrategien}
Kurz das kleine Java Projekt vorstellen mit den Beispieldaten die gespeichert und geladen
werden sollen. Erwähnen, dass JMH zum messen der Effizienz verwendet wurde.\\

Faktoren:\\
\begin{itemize}
    \item Chunk size
    \item Veränderungen
\end{itemize}

\section{Ladesysteme}
Erstmal Theorie vom Laden der Spieldaten erarbeiten. Danach mit Spielkonzept
von 2.1 testen und Laufzeiten messen.
%--------------------------------------------------------------------------


%--------------------------------------------------------------------------
\subsection{Nearby Loading}
Nur die Daten laden, die in der Nähe vom Spieler sind (über Render-Distance)\\
Chunk System dafür sehr vorteilhaft für effizienteres Laden, statt durch alle Elemente
immer zu iterieren.
%--------------------------------------------------------------------------


%--------------------------------------------------------------------------
\subsection{Lazy Loading}
Weiß nicht, ob das der richtige Begriff dafür ist, aber hier soll über das prozedulare 
Laden der Daten geschrieben werden (nicht alles auf einmal laden, sondern erst mal nur das
nötigste und dann Stück für Stück den Rest laden). 
%--------------------------------------------------------------------------


%--------------------------------------------------------------------------
\subsection{Effizienz der Ladestrategien}
Laufzeiten vom Laden messen. Vielleicht läuft das Laden mit manchen Speicherstrategien 
besser als mit anderen. Außerdem ist die Performance nicht nur an Zeit zu messen, sondern
an Ressourcen, die vom Computer benötigt werden. Vielleicht gibt es Lade Strategien, die
leicht gewichtet laufen können.
\chapter{Fazit}\label{ch:fazit}
Die Ergebnisse dieser Arbeit zeigen, dass Speicher- und Ladesysteme durchaus sehr komplex sein können. Es gibt viele Möglichkeiten beim Zusammensetzen und Erstellen dieser Systeme und auch viele Herausforderungen, die beachtet werden müssen. Auch in Videospielen verwenden viele Spiele sehr unterschiedliche Ansätze in dieser Thematik. 

Für Effizienz bei Speicher- und Ladesystemen ist im Kapitel \ref{ch:algorithmen} zu sehen, dass Strategien mit binärer Serialisierung am schnellsten sind. Mithilfe des Protocol Buffers von Google lässt sich diese auch einfach umsetzen. Außerdem kann die Datenstruktur noch ausgebaut und angepasst werden. Mittels einfacher Konvertierung zu dem JSON-Format lassen sich auch bei Bedarf die gespeicherten Binärdaten lesen. Für ein anpassungsfähiges System sind Strategien mit dynamischen Chunk-Systemen äußerst vielversprechend. Da diese jedoch aufwändig zu entwickeln sind, sollten bei Spielen mit Daten, die sich von der Menge nicht stetig groß verändern, eher eine Strategie mit statischen Chunk-System verwendet werden. Dieses System lässt sich mit deutlich geringeren Aufwand implementieren. Im Allgemeinen empfiehlt es sich jedoch, mit möglichst großen Chunks zu arbeiten, da diese die Spieldaten schneller verwalten können. Ein Komprimierungsalgorithmus wie \ac{gzip} ist für die Effizienz nicht notwendig. Wenn jedoch die Kompaktheit der Daten wichtig ist, dann verschlechtern Strategien, die Komprimierungsalgorithmen verwenden, die Laufzeit nicht stark. 
%Diese Strategien sind nicht ineffizient, jedoch sind im Vergleich Strategien ohne Komprimierungsalgorithmus schneller.

Bei der Verwendung der untersuchten Game Engines ist im Kapitel \ref{ch:gameengines} aufgefallen, dass diese begrenzte Unterstützung für Entwickler bei Speicher- und Ladesystemen anbieten. Sie bieten nur Funktionen für das Serialisieren der Spieldaten, dabei in den meisten Fällen nur in das JSON-Format. Diese Funktionen sind stark optimiert worden, jedoch muss vieles vom Speicher- und Ladesystem von den Entwicklern selber erstellt werden. Über Extensions lassen sich jedoch Systeme verwenden, die viel Arbeit für den Entwickler abnehmen und trotzdem eine gute Laufzeit anbieten. Diese sind eine gute Option für Entwickler, die wenig Zeit für das Speicher- und Ladesystem verbringen wollen.

In der Videospieleindustrie scheint es, nach Kapitel \ref{ch:videospiele} zu beurteilen, keinen Standard zu geben. Diese Thematik scheint noch recht unerforscht zu sein, da jeder Entwickler andere Strategien verwendet. Bei dem Spiel Minecraft schien das Speicher- und Ladesystem jedoch sehr durchdacht zu sein. Dieses System wurde auch über Jahre weiter verbessert, was zur allgemeinen Verbesserung des Spieles beigetragen hat. Hier konnten zum Beispiel dank Verbesserung der Effizienz Spielwelten mit höheren Grenzen eingeführt werden. Bei dem Spiel Factorio scheint das Speichern und Laden des Spielstandes komplex geworden zu sein. Hier gibt es bei dem Speicher- und Ladeprozess viele wiederholende Prozesse und es werden beim Speichern und Laden einige Spielobjekte neu hinzugefügt, bearbeitet oder sogar gelöscht. Das Gegenteil bei der Komplexität ist der Ansatz von Stardew Valley. Hier wurde das Speicher- und Ladesystem sehr schlicht gehalten, dafür werden die Daten aber auch nicht so häufig gesichert. 
\todo{Ausblick}
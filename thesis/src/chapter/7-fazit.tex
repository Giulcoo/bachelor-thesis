\chapter{Fazit}\label{ch:fazit}
\todo{Kleine Einleitung}
Für Speicher- und Ladesystemen ist im Kapitel \ref{ch:algorithmen} zu sehen, dass Strategien mit binärer Serialisierung am schnellsten sind. Mithilfe des Protocol Buffer von Google lässt sich diese auch einfach umsetzen. Außerdem kann die Datenstruktur noch ausgebaut und angepasst werden. Mittels einfacher Konvertierung zu dem JSON-Format lassen sich auch bei Bedarf die gespeicherten Binärdaten lesen. Für ein anpassungsfähiges System sind Strategien mit dynamischen Chunk-Systemen äußerst vielversprechend. Da diese jedoch aufwändig zu entwickeln sind, sollten bei Spielen mit Daten, die sich von der Menge nicht stetig groß verändern, eher eine Strategie mit statischen Chunk-System verwendet werden. Dieses System lässt sich mit deutlich geringeren Aufwand aufbauen. Im Allgemeinen empfiehlt es sich jedoch mit möglichst großen Chunks zu arbeiten, da diese die Spieldaten schneller managen\todo{synonym} können. Einen Komprimierungsalgorithmus, wie \ac{gzip}, ist für die Effizienz nicht nötig. Wenn jedoch die Kompaktheit der Daten wichtig ist, dann verschlechtern Strategien, die Komprimierungsalgorithmen verwenden, die Laufzeit nicht stark. Diese Strategien sind nicht uneffizient, jedoch sind im Vergleich Strategien ohne Komprimierungsalgorithmus schneller.

Bei der Verwendung von den meisten modernen Game Engines ist im Kapitel \ref{ch:gameengines} aufgefallen, das diese wenig Unterstützung für Entwickler bei Speicher- und Ladesysteme anbieten. Die meisten Game Engines bieten nur Funktionen für das Serialisieren der Spieldaten. Dabei auch in den meisten Fällen nur in das JSON-Format. Diese Funktionen sind auch stark optimiert worden, jedoch muss vieles vom Speicher- und Ladesystem von den Entwickler selber erstellt werden. Über Extensions lassen sich jedoch Systeme verwenden, die sehr viel Arbeit für den Entwickler abnehmen und trotzdem eine gute Laufzeit anbieten. Diese sind eine gute Option für Entwickler, die wenig Zeit für das Speicher- und Ladesystem verschwenden\todo{synonym} wollen.

In der Videospieleindustrie scheint es nach Kapitel \ref{ch:videospiele} zu beurteilen keinen Standard zu geben. Diese Thematik scheint noch recht unerforscht zu sein, da jeder Entwickler andere Strategien verwendet. Bei dem Spiel Minecraft schien das Speicher- und Ladesystem jedoch sehr durchdacht zu sein. Dieses System wurde auch über Jahre weiter verbessert, was zu der allgemeinen Verbesserung des Spieles gesorgt hat. Hier konnten zum Beispiel dank Verbesserung der Effizienz Spielwelten mit höheren Grenzen eingeführt werden. Bei dem Spiel Factorio scheint das Speichern und Laden des Spielstandes etwas zu komplex geworden zu sein. Hier gibt es bei dem Speicher- und Ladeprozess viele wiederholende Prozesse und es werden beim Speichern und Laden einige Spielobjekte neu hinzugefügt, bearbeitet oder sogar gelöscht. Das Gegenteil bei der Komplexität ist der Ansatz von Stardew Valley. Hier wurde das Speicher- und Ladesystem sehr schlicht gehalten, dafür werden die Daten aber auch nicht so häufig gesichert. 

% \begin{itemize}
%     \item Binäre Serialisierung als Format am schnellsten
%     \begin{itemize}
%         \item Mit Protocol Buffer auch einfach zu benutzen, modifizieren und in wenigen Schritten lesbar
%     \end{itemize} 
%     \item Dynamische Chunk-Systeme sehr anpassungsfähig, aber aufwendiger zu Entwickeln
%     \item Statische Chunk-Systeme schneller umzusetzen und ausreichend bei Spielwelten, wo Daten sich nicht dauert ändern
%     \item Je größer die Chunks, desto schneller sind die Daten zu managen, speichern und laden
%     \item Komprimierungsalgorithmen für Effizienz nicht optimal, schaden aber nicht, wenn Daten kompakt sein müssen 
%     \begin{itemize}
%         \item Trotzdem nicht sehr uneffizient, aber es gibt ein paar schnellere Strategien ohne
%         \item Gut, wenn Daten kompakt sein sollen und trotzdem keine große Verlangsamung
%     \end{itemize}
%     \item Meisten modernen Game Engines bieten wenig Unterstützung beim Entwickeln eines Speicher- und Ladesystems
%     \begin{itemize}
%         \item Viele Game Engines bieten nur Funktionen zum binären Serialisieren oder Serialisieren zu JSON an
%         \item Diese sind zwar stark optimiert und einfach zu verwenden, jedoch muss vieles vom Speicher- und Ladesystem von den Entwicklern selber erstellt werden.  
%         \item Die meisten haben jedoch Extensions, die dabei viel helfen können
%     \end{itemize}
%     \item Bei der Spieleindustrie scheint kein Standard zu existieren, jeder arbeitet unterschiedlich
%     \item Minecraft scheint ein sehr durchdachtes System zu haben, welches sich über die Jahre noch weiter verbessert hat und damit bessere Performance erreicht hat (z.B. Erhöhung der Spielwelt)
%     \item Factorio scheint das etwas überkompliziert anzugehen und hat sehr viele wiederholende Schritte
%     \item Stardew Valley geht das ganze noch viel einfacher an, dafür wird aber auch nur alle 14min das Spiel gespeichert 
% \end{itemize}
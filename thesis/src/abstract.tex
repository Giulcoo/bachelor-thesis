\chapter*{Zusammenfassung}

% Inhaltsverzeichnis und Kopfzeile
\addcontentsline{toc}{chapter}{Zusammenfassung}
\markboth{Zusammenfassung}{Zusammenfassung}

In dieser Arbeit werden Speicher- und Ladesysteme von Videospielen betrachtet. Diese Systeme sollen beim Start eines Spieles den Spielstand und während der Spielphase noch weitere benötigten Daten des Spielstandes laden und stets den aktuellen Stand absichern können. Dies sollte möglichst effizient passieren, damit die Systeme die Erfahrung für Spieler nicht stören und im Hintergrund der Anwendung laufen können. Dafür werden verschiedene Methoden des Speicherns und Ladens der Spielobjekte vorgestellt und mittels eines Testszenario und Laufzeitmessungen verglichen. Auf dem Testszenario werden verschiedene Basisfunktionen eines Speicher- und Ladesystems getestet, damit die Stärken und Schwachstellen von jeder Strategie bestimmt werden können. Außerdem wird betrachtet, ob in den populären modernen Game Engines Funktionen eingebaut sind oder Techniken verwendet werden, die das Aufbauen eines Speicher- und Ladesystems erleichtern oder größtenteils übernehmen. Des Weiteren wird untersucht, welche Strategien in der Spieleindustrie verwendet werden. Dazu werden verschiedene populäre Videospiele und deren Strategien des Speicherns und Ladens eines Spielstandes betrachtet.

\chapter*{Abkürzungsverzeichnis}

% Inhaltsverzeichnis und Kopfzeile
\addcontentsline{toc}{chapter}{Abkürzungsverzeichnis}
\markboth{Abkürzungsverzeichnis}{Abkürzungsverzeichnis}

% Auf diese Weise kann der Plural von unbekannten Wörtern definiert werden (für \acp{..})
\acrodefplural{ha}[HAs]{Hausaufgaben}
\acrodefplural{rq}[RQs]{Forschungsfragen}

\begin{acronym}[XXXXXX] % Anzahl der X gibt an, welche Breite das längste Acronym hat
    % Es werden nur Akronyme übernommen, die auch verwendet werden    
    \acro{api}[API]{Application Programming Interface}
    \acro{bson}[BSON]{Binary JSON}
    \acro{css}[CSS]{Cascading Style Sheet}
    \acro{gui}[GUI]{Graphical User Interface}
    \acro{ha}[HA]{Hausaufgabe}
    \acro{html}[HTML]{Hypertext Markup Language}
    \acro{http}[HTTP]{Hypertext Transfer Protocol}
    \acro{https}[HTTPS]{Hypertext Transfer Protocol Secure}
    \acro{id}[ID]{Identifier}
    \acro{ide}[IDE]{Integrated Development Environment}
    \acro{ip}[IP]{Internet Protocol}
    \acro{json}[JSON]{JavaScript Object Notation}
    \acro{jvm}[JVM]{Java Virtual Machine}
    \acro{mca}[MCA]{Minecraft Anvil}
    \acro{nbt}[NBT]{Named Binary Tag}
    \acro{rest}[REST]{Representational State Transfer}
    \acro{rq}[RQ]{Forschungsfrage}
    \acro{sdk}[SDK]{Software Development Kit}
    \acro{ssh}[SSH]{Secure Shell}
    \acro{ui}[UI]{User Interface}
    \acro{url}[URL]{Uniform Resource Locator}
    \acro{xml}[XML]{Extensible Markup Language}
    
    \vspace{\parskip}
    
    \acro{gzip}[Gzip]{GNU zip}
    \acro{jmh}[JMH]{Java Microbenchmark Harness}
    \acro{lz77}[LZ77]{Lempel-Ziv 77}
    \acro{protoc}[protoc]{Protocol Buffer Compiler}
    \acro{vts}[VTS]{Version tolerant serialization}
    \acro{xsd}[XSD]{XML Schema Definition}
    \acrodefplural{xsd}[XSDs]{XML Schema Definitions}
    
    \vspace{\parskip}

    \acro{ops}[ops/s]{Operations per second}

    \vspace{\parskip}

    \acro{bzw}[bzw.]{beziehungsweise}
    \acro{dh}[d.h.]{das heißt}
    \acro{etc}[etc.]{et cetera}
    \acro{idR}[i.d.R.]{in der Regel}
    \acro{oBdA}[o.B.d.A.]{ohne Beschränkung der Allgemeinheit}
    \acro{sic}[sic]{sic erat scriptum}
    \acro{usw}[usw.]{und so weiter}
    \acro{uU}[u.U.]{unter Umständen}
    \acro{vgl}[vgl.]{vergleiche}
    \acro{zB}[z.B.]{zum Beispiel}

    \vspace{\parskip}

    \acro{mod}[Mod]{Video Game modification}
    \acrodefplural{mod}[Mods]{Video Game modifications}
    \acro{npc}[NPC]{Non-Player Character}
    \acrodefplural{npc}[NPCs]{Non-Player Characters}
\end{acronym}

% Select input encodung, usually utf8 is the best choice, on windows, \usepackage[latin1]{inputenc} maybe required
\usepackage[utf8]{inputenc}
\usepackage[T1]{fontenc}
\usepackage[ngerman]{babel}
\usepackage{csquotes}
\usepackage{xcolor}

\MakeOuterQuote{"} % Damit ist es möglich, " " zu verwenden ohne Umlaut zu erzeugen
\defaulthyphenchar=127 % Dadurch werden auch Wörter mit Bindestrich getrennt, die schon Bindestriche enthalten.

% geometry
\usepackage[bindingoffset=1cm, left=2.5cm, right=2.5cm, top=2.5cm, bottom=2.5cm]{geometry}

% Headline
\usepackage{fancyhdr}
\pagestyle{fancy}
\renewcommand{\chaptermark}[1]{\markboth{\thechapter\ #1}{}}
\lhead{\leftmark} \rhead{\thepage}
\cfoot{}
\fancypagestyle{plain}{}

\RedeclareSectionCommand[beforeskip=1.5cm,afterskip=1cm]{chapter}

% Colors
\usepackage{color}
\usepackage{colortbl}

% Tables
\usepackage{tabularx}
\usepackage{multirow}
\setlength{\tabcolsep}{4pt}

% Drawing graphs etc.
\usepackage{pgf}
\usepackage{tikz}
\usetikzlibrary{arrows,automata}

% Footnotes
\usepackage{footmisc}

\usepackage{xspace}
\newcommand{\sic}{[\acs{sic}]\xspace}

% math
\usepackage{amsmath}
\usepackage{siunitx}

% lists
\usepackage{paralist}

% Figures
\usepackage{graphicx, wrapfig}
\usepackage{svg}

% Hyperlinks
\usepackage[hyphens]{url}
\usepackage{hyperref}
\hypersetup{colorlinks, citecolor=black, linkcolor=black, urlcolor=black}

% Minted
\usepackage[chapter]{minted}
%\usemintedstyle{xcode}
\setminted{frame=single,tabsize=2,linenos,autogobble}

\newmintinline[code]{text}{breaklines}

\newminted[mdcodeblock]{md}{autogobble,frame=none,linenos=false,breaklines}

% list of abbreviations
\usepackage[printonlyused]{acronym}

% Set line pitch
\usepackage{setspace}
\onehalfspacing              % anderthalbzeilig (oder auch \doublespace)

%fancyBox
%\usepackage{fancybox}

% Layout corrections (Schusterjungen)
\clubpenalty = 10000
% Layout corrections (Hurenkinder)
\widowpenalty = 10000
\displaywidowpenalty = 10000

% Figures
\usepackage{caption}
\usepackage[hypcap=true,labelformat=simple]{subcaption}
\renewcommand{\thesubfigure}{(\alph{subfigure})}

% Tables
\usepackage{booktabs}

% Todos
\usepackage{todonotes}

% Directory Tree
\usepackage{dirtree}
\DTsetlength{1em}{2em}{0.2em}{0.4pt}{2pt}

% Frequently used column types
\newcolumntype{C}[1]{>{\centering\arraybackslash}p{#1}} % centering column type with fixed width
\newcolumntype{R}[1]{>{\raggedleft\arraybackslash}p{#1}} % right aligned column type with fixed width
\newcolumntype{L}[1]{>{\raggedright\arraybackslash}p{#1}} % left aligned column type with fixed width

% Shortcuts for referencing floats:
\newcommand{\fig}[1]{\figurename~\ref{#1}} %shortcut for a figure reference
\newcommand{\tab}[1]{Table~\ref{#1}} %shortcut for a table reference
\newcommand{\eq}[1]{(\ref{#1})} %shortcut for an equation reference
\newcommand{\lst}[1]{Listing~\ref{#1}} %shortcut for a listing reference
\newcommand{\sect}[1]{Section~\ref{#1}} %shortcut for a Section reference
\newcommand{\br}[0]{\hspace{0cm}\\}

% Custom commands
\newcommand{\wichtig}[1]{\textbf{\textcolor{red}{#1}}}
